\documentclass{beamer}

\usepackage[T1]{fontenc}
\usepackage[utf8]{inputenc}
\usepackage[ngerman]{babel}

\usetheme{Madrid}
\usecolortheme{crane}

\usepackage{listings}
\lstset{%
	language=Python,
	xleftmargin=1cm,
	xrightmargin=1cm,
	backgroundcolor=\color{lightgray},
	keepspaces=true
}

\title{HowTo:Python}
\author{Paua-App // Louis Kobras}

\begin{document}
\begin{frame}
	\maketitle
\end{frame}
\begin{frame}[fragile]
	\frametitle{Setup und Stuff}
	\begin{itemize}
		\item \textbf{Linux:} *tada*
		\item \textbf{Windows:} Setup von \texttt{python.org} herunterladen und ausführen
		\item Unter Windoof muss die PATH Variable gesetzt werden
	\end{itemize}
	\pause
	\begin{lstlisting}
		python myscript.py
		.
		.
		.
		python
		>>>
	\end{lstlisting}
\end{frame}
\begin{frame}[fragile]
	\frametitle{Syntax}
	\begin{itemize}
		\item ``runnable pseudo code''
		\item Wie Java, ohne Klammern
		\item Indentation-empfindlich
		\item ASCII-Encoding
		\item Kommentar: \#
	\end{itemize}
	\pause
		\begin{lstlisting}
			# This is a comment
			print "Hello World."
			name = raw_input("What is your name?")
			print "Hello, %s" % name
		\end{lstlisting}
\end{frame}


\begin{frame}[fragile]
	\frametitle{Kontrollstrukturen}
	\begin{itemize}
		\item if-(elif)*-(else)?\footnote{Man beachte die RegEx-Syntax}
		\item while
		\item foreach
	\end{itemize}
	\pause
	\begin{lstlisting}
		if condition:
    		doStuff()
		elif foo:
		    bar()
		elif spam:
		    eggs()
		else:
		    # todo
	\end{lstlisting}
\end{frame}

\begin{frame}[fragile]
	\frametitle{Daten}
	\begin{itemize}
		\item Everything is data (even functions)
		\item dynamisch getypt
		\item Funktionen können Tupel zurückgeben
		\item Default-Parameter möglich
	\end{itemize}
	\pause
	\begin{lstlisting}
		# alles gueltige Deklarationen
		a = "String"
		b = 0
		c = 0.7
		d = a
		a += 1
	\end{lstlisting}
\end{frame}

\frame{
	\frametitle{Mathe in Python}
	\begin{center}
		\begin{Huge}
			\#Demo
		\end{Huge}
	\end{center}
}

\begin{frame}
	\frametitle{OOP 1}
	\begin{itemize}
		\item Klassen als Schablonen
		\item Magic Methods für Objektverhalten
		\item Private Methoden und Felder möglich, jedoch ungenutzt
	\end{itemize}
	\pause
	Magic Methods:
	\begin{itemize}
		\item init (Konstruktor)
		\item add, sub, mul, div \dots (Arithmetik)
		\item eq, ne, lt, gt, le, ge (Vergleich)
		\item pos, neg, floor, ceil \dots (Numerik)
		\item viele, viele mehr (googlen)
	\end{itemize}
\end{frame}

\begin{frame}[fragile]
	\frametitle{OOP 2}
	\begin{lstlisting}
		class foo:
			lst = []
		    def __init__(self, lst):
		        self.lst = lst
		    def __add__(self, lst):
		        self.lst.append(lst)
		        return self.lst
	\end{lstlisting}
	Konstruktoraufruf: bar = foo([a,b,c])\\
	Übergibt die Liste [a,b,c] an init
\end{frame}

\begin{frame}[fragile]
	\frametitle{OOP 3 - Vererbung}
	\begin{itemize}
		\item Mutterklassen in Klammern
		\item implizites Override von Methoden und Feldern
		\item transitiv
		\item aufrufen von Eigenschaften der Superklasse mit \texttt{super().spam}
	\end{itemize}
	\pause
	\begin{lstlisting}
	class animal:
	    ...
	    def bark():
	        print "woof"
	    	
	class wolf(animal):
	    ...
	    
	    
	>>> wolf.bark()
	"woof"
	\end{lstlisting}
\end{frame}

\begin{frame}[fragile]
	\frametitle{Funktionale Programmierung}
	\begin{itemize}
		\item Rekursion und Tail Recursion implementierbar
		\item Funktionen höherer Ordnung vorhanden und baubar
		\item \texttt{lambda}-Funktion:
	\end{itemize}
	\begin{lstlisting}
	\pause
		>>> def square(lst):
		...    o = []
		...    for e in lst:
		...        a = lambda x: x*x
		...        o.append(a(e))
		...    return o
		>>> square([1,2,3])
		[1, 4, 9]
	\end{lstlisting}
	$\Rightarrow$ \texttt{lambda}-Funktion an den Buchstaben 'a' gebunden
\end{frame}

\begin{frame}[fragile]
	\frametitle{Module}
	\begin{itemize}
		\item Importe mit \texttt{import >>module<<} oder \texttt{from >>module<< import >>function<<}
		\item Importe können benannt werden (\texttt{... as >>nick<<})
		\item Bedingte Importe möglich:
	\end{itemize}
	\pause
	\begin{lstlisting}
		from math import factorial as fac
		if os.name == "WinDoof":
		    import thatOneLib as lib
		elif os.name == "Linux":
		    import thatOtherLib as lib
		else:
		    import thatStandartLib as lib
	\end{lstlisting}
\end{frame}

\begin{frame}
	\frametitle{Grafik Stuff}
	\begin{itemize}
		\item drölf verschiedene Frameworks
		\item Auswahl je nach gewünschter Funktionalität und Zielplattform(en)
	\end{itemize}
\end{frame}

\begin{frame}
	\frametitle{About - Random Facts and Stuff}
	\begin{itemize}
		\item PEPs
		\item Zen of Python (\texttt{import this})
		\item BDFL
	\end{itemize}
\end{frame}
	
\end{document}