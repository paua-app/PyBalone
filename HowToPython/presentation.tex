\documentclass[handout]{beamer}

\usepackage[T1]{fontenc}
\usepackage[utf8]{inputenc}
\usepackage[ngerman]{babel}

\usetheme{Luebeck}
\usecolortheme{crane}

\usepackage{listings}
\lstset{%
	language=Python,
	xleftmargin=1cm,
	xrightmargin=1cm,
	backgroundcolor=\color{lightgray},
	keepspaces=true
}


\begin{document}
\begin{frame}[fragile]
	\frametitle{Syntax}
	\begin{itemize}
		\item ``runnable pseudo code''
		\item Wie Java, ohne Klammern
		\item Indentation-empfindlich
		\item ASCII-Encoding
		\item Kommentar: \#
	\end{itemize}
		\begin{lstlisting}
			# This is a comment
			print "Hello World."
			name = raw_input("What is your name?")
			print "Hello, %s" % name
		\end{lstlisting}
\end{frame}


\begin{frame}[fragile]
	\frametitle{Kontrollstrukturen}
	\begin{itemize}
		\item if-(elif)*-(else)?
		\item while
		\item foreach
	\end{itemize}
	\begin{lstlisting}
		if condition:
    		doStuff()
		elif foo:
		    bar()
		elif spam:
		    eggs()
		else:
		    # todo
	\end{lstlisting}
\end{frame}

\begin{frame}[fragile]
	\frametitle{Daten}
	\begin{itemize}
		\item Everything is data (even functions)
		\item dynamisch getypt
		\item Funktionen können Tupel zurückgeben
		\item Default-Parameter möglich
	\end{itemize}
	\begin{lstlisting}
		# alles gueltige Deklarationen
		a = "String"
		b = 0
		c = 0.7
		d = a
		a += 1
	\end{lstlisting}
\end{frame}

\frame{
	\frametitle{Mathe in Python}
	\begin{center}
		\begin{Huge}
			\#Demo
		\end{Huge}
	\end{center}
}

\begin{frame}
	\frametitle{OOP 1}
	\begin{itemize}
		\item Klassen als Schablonen
		\item Magic Methods für Objektverhalten
	\end{itemize}
	Magic Methods:
	\begin{itemize}
		\item init (Konstruktor)
		\item add, sub, mul, div \dots (Arithmetik)
		\item eq, ne, lt, gt, le, ge (Vergleich)
		\item pos, neg, floor, ceil \dots (Numerik)
		\item viele, viele mehr (googlen)
	\end{itemize}
\end{frame}

\begin{frame}[fragile]
	\frametitle{OOP 2}
	\begin{lstlisting}
		class foo:
			lst = []
		    def __init__(self, lst):
		        self.lst = lst
		    def __add__(self, lst):
		        self.lst.append(lst)
		        return self.lst
	\end{lstlisting}
	Konstruktoraufruf: bar = foo([a,b,c])\\
	Übergibt die Liste [a,b,c] an init
\end{frame}

\begin{frame}[fragile]
	\frametitle{OOP 3 - Vererbung}
\end{frame}

\begin{frame}[fragile]
	\frametitle{Funktionale Programmierung}
	\begin{itemize}
		\item Rekursion und Tail Recursion implementierbar
		\item Funktionen höherer Ordnung vorhanden und baubar
		\item \texttt{lambda}-Funktion:
	\end{itemize}
	\begin{lstlisting}
		>>> def square(lst):
		...    o = []
		...    for e in lst:
		...        a = lambda x: x*x
		...        o.append(a(e))
		...    return o
		>>> square([1,2,3])
		[1, 4, 9]
	\end{lstlisting}
\end{frame}

\begin{frame}
	\frametitle{Pakete und Module}
	\begin{itemize}
		\item Importe mit \texttt{import >>module<<} oder \texttt{from >>module<< import >>function<<}
		\item Importe können benannt werden (\texttt{... as >>nick<<})
		\item Bedingte Importe möglich
	\end{itemize}
\end{frame}

\begin{frame}
	\frametitle{Grafik Stuff}
	\begin{itemize}
		\item drölf verschiedene Frameworks
		\item Auswahl je nach gewünschter Funktionalität und Zielplattform(en)
	\end{itemize}
\end{frame}
	
\end{document}